\documentclass[12pt,a4paper]{scrreprt}
\usepackage[utf8]{inputenc}
\usepackage[T1]{fontenc}
\usepackage[ngerman]{babel}
\usepackage[a4paper,scale=0.8]{geometry}
 

\usepackage{paralist}
\usepackage{enumitem} 

%\usepackage {picins}
\usepackage{color}
\usepackage{float}

% -- Mathe
%\usepackage{gensymb}
%fancyhdr, lastpage, booktabs, xy
\usepackage{graphicx}
\usepackage{amsmath}
\usepackage{amsfonts}
\usepackage{amssymb}

\usepackage{amsthm}

\usepackage{hyperref}


% -- Malen
\usepackage{tikz}

\usetikzlibrary{patterns, decorations.pathreplacing, decorations.pathmorphing, arrows}

\usepackage{units}

\numberwithin{equation}{section}

\author{Philipp Beck, Daniel Winkle}

\subtitle{Wintersemester 17/18}
\title{Klassische Theorie partieller Differentialgleichungen}


%Makros
\newcommand{\bi}[1]{\textbf{\textit{#1}}}
\newcommand{\changefont}[3]{\fontfamily{#1} \fontseries{#2} \fontshape{#3} \selectfont}


%Befehle für bestimmte mathematische Symbole
\newcommand{\td}[1]{ \ \text{\rmfamily d} #1}
\renewcommand{\div}{\operatorname{div}}
\newcommand{\rot}{\operatorname{rot}}
\newcommand{\grad}{\operatorname{grad}}
\newcommand{\diag}{\operatorname{diag}}
\renewcommand{\i}{\textrm{i}}
\newcommand{\Id}{\operatorname{Id}}
\newcommand{\supp}{\operatorname{supp}}

%Mengensymbole
\newcommand{\N}{\mathbb{N}}
\newcommand{\Z}{\mathbb{Z}}
\newcommand{\Zn}[1]{\mathbb{Z}/#1\mathbb{Z}}
\newcommand{\Q}{\mathbb{Q}}
\newcommand{\R}{\mathbb{R}}
\newcommand{\C}{\mathbb{C}}




\renewcommand\qedsymbol{$\blacksquare$}

\swapnumbers

\newtheoremstyle{satz}% name
{15pt}% Space above
{5pt}% Space below
{\itshape}% Body font
{}% Indent amount: Indent amount: empty = no indent, \parindent = normal paragraph indent
{\bfseries}% Theorem head font
{:}% Punctuation after theorem head
{\newline}% Space after theorem head: Space after theorem head: { } = normal interword space; \newline = linebreak
{}% Theorem head spec (can be left empty, meaning `normal')

\newtheoremstyle{def}% name
{15pt}% Space above
{5pt}% Space below
{}% Body font
{}% Indent amount: Indent amount: empty = no indent, \parindent = normal paragraph indent
{\bfseries}% Theorem head font
{:}% Punctuation after theorem head
{\newline}% Space after theorem head: Space after theorem head: { } = normal interword space; \newline = linebreak
{}% Theorem head spec (can be left empty, meaning `normal')

\newtheoremstyle{exercise}% name
{15pt}% Space above
{5pt}% Space below
{}% Body font
{}% Indent amount: Indent amount: empty = no indent, \parindent = normal paragraph indent
{\bfseries}% Theorem head font
{:}% Punctuation after theorem head
{0.5em}% Space after theorem head: Space after theorem head: { } = normal interword space; \newline = linebreak
{}% Theorem head spec (can be left empty, meaning `normal')





\newcommand{\thistheoremname}{}
\theoremstyle{satz}
%Satz
\newtheorem{sz}{Satz}[chapter]
%\renewcommand{\thesz}{\arabic{\thechapter.\arabic{sz}}
%Lemma
\newtheorem{lemma}[sz]{Lemma}
\newtheorem{gsz}[sz]{\thistheoremname}
\newtheorem*{gthm_no_num}{\thistheoremname}


\theoremstyle{def}
\newtheorem{df}[sz]{Definition}
\newtheorem{gd}[sz]{\thistheoremname}

\newtheorem*{gdf_no_num}{\thistheoremname}


\theoremstyle{exercise}
\newtheorem{exe}{Aufgabe}[section]
\newtheorem{loes}{Lösung}[section]

%Für Beispiele , Bemerkungen
%Also alles was wie eine Definition aussehen soll, aber keine ist.
%\begin{genericdf}{(Hier den Text einfügen)}
%	Inhalt...
%\end{genericdf}
\newenvironment{genericdf}[1]{\renewcommand{\thistheoremname}{#1}\begin{gd}}{\end{gd}}
%Dasselbe ohne Nummerierung
\newenvironment{generidf_no_num}[1]{\renewcommand{\thistheoremname}{#1}\begin{gdf_no_num}}{\end{gdf_no_num}}

%Hier für Sätze
\newenvironment{genericthm}[1]{\renewcommand{\thistheoremname}{#1}\begin{gsz}}{\end{gsz}}
%Analog zu Definitionen
\newenvironment{genericthm_no_num}[1]{\renewcommand{\thistheoremname}{#1}\begin{gthm_no_num}}{\end{gthm_no_num}}
\parindent0pt
%\parskip1ex


%Ändere Chapter
\setkomafont{chapter}{\changefont{pnc}{b}{sc}\Huge\bfseries}

%Ändere Section
\setkomafont{section}{\changefont{pnc}{b}{sc}\large\bfseries}

%----Dokument----

\begin{document}
%\changefont{pnc}{m}{n}
\maketitle
\chapter{Einführung}
GITHUB SUCKS
\section{Partielle Differentialgleichungen aus der Physik}
\begin{genericdf}{Elektrostatik}
Sei $\rho  :  \R^3 \to \R$ eine Ladungsverteilung bzw. Dichte
und $E : \R^3 \to \R^3$ ein elektrisches Feld.
Aus der Physik ist der Zusammenhang
\begin{align*}
\underbrace{\int_\Omega \rho \td{x}}_{\text{Ladung in $\Omega$}}
=
\underbrace{\int_{\partial\Omega} n \ E \td{\sigma}}_{\text{Aus $\Omega$ ausfließendes Feld}}
\stackrel{\text{Gauß-O.}}{=}
\int_\Omega \div E \td{x}
\end{align*}
für \glqq alle\grqq~Gebiete $\Omega \subseteq \R^3$ bekannt.
Hierbei ist $n$ der aus $\Omega$ hinauszeigende Normaleneinheitsvektor.
Da $\Omega$ beliebig ist, folgt
\begin{align*}
\rho = \div E 
:= \partial_{x_1} E_1 + \partial_{x_2} E_2 + \partial_{x_3} E_3.
\end{align*}
Außerdem ist ein elektrisches Feld wirbelfrei, womit
\begin{align*}
\rot E := \nabla \times E
=
\begin{pmatrix}
\partial_{x_1} \\
\partial_{x_2} \\
\partial_{x_3}
\end{pmatrix} 
\times
\begin{pmatrix}
E_1 \\ E_2 \\ E_3
\end{pmatrix}
= 
\begin{pmatrix}
\partial_{x_2} E_3 - \partial_{x_3} E_2\\
\partial_{x_3} E_1 - \partial_{x_1} E_3\\
\partial_{x_1} E_2 - \partial_{x_2} E_1
\end{pmatrix}
= 
0
\end{align*}
gilt. Damit besitzt ein $E$ ein Potential $\varphi : \R^3 \to \R$, d.h. es gilt
\begin{align*}
E = \grad \varphi := \nabla \varphi
:=
\begin{pmatrix}
\partial_{x_1} \varphi \\
\partial_{x_2} \varphi \\
\partial_{x_1} \varphi 
\end{pmatrix} 
\Rightarrow
\Delta \varphi = \rho.
\end{align*}
Dabei bezeichnet
\begin{align*}
\Delta = \partial_{x_1}^2 + \partial_{x_2}^2 + \partial_{x_3}^2
\end{align*}
den \textit{Laplace-Operator}.
Nun bezeichnet man
\begin{align*}
\Delta \varphi &= \rho\\
\Delta \varphi &= 0
\end{align*}
als \textit{Poisson-bzw. Laplacegleichung}.
\end{genericdf}

\begin{genericdf}{Wärmeleitung}
Sei $\nu:[0, \infty) \times \R^3 \to \R$ eine Temperaturverteilung.
Dann ist die Wärmeenergie durch
\begin{align*}
\int \limits_\Omega c \cdot \nu(t,x) \td{x}
\end{align*}
für eine Materialkonstante $c$ gegeben.
Die Energieänderung erhält man nun durch
\begin{align*}
\int \limits_\Omega c \cdot \nu(t_1,x) \td{x}
&-
\int \limits_\Omega c \cdot \nu(t_2,x) \td{x}
\stackrel{\text{Physik}}{=}
\int \limits_{t_1}^{t_2} 
\underbrace{\int \limits _{\partial \Omega} 
\alpha \ \nabla \nu(t,x) \cdot n \td{x}}_{\text{Wärmefluss aus $\Omega$}} \td{t}\\
&\stackrel{\text{Gauß-O.}}{=}
\int \limits_{t_1}^{t_2}
\int \limits_\Omega 
\nabla \cdot ( \alpha \ \nabla \nu(t,x)) \td{x} \td{t}.
\end{align*}
Da $t_1,t_2 $ und $\Omega$ beliebig war, folgt
\begin{align*}
\int \limits_{\Omega}
c \ \partial_t \nu(t,x)  \td{x}
= 
\int \limits_{\Omega}
\nabla \cdot ( \alpha \ \nabla \nu(t,x)) \td{x}
\end{align*}
und
\begin{align*}
\partial_t \nu(t,x) = \frac{1}{c} \ \nabla \cdot ( \alpha \ \nabla \nu(t,x)).
\end{align*}
Es folgt
\begin{align*}
\partial_t \nu(t,x) = \frac{\alpha}{c} \Delta\nu(t,x),
\end{align*}
falls $\alpha$ eine Konstante ist.
Diese Differentialgleichung nennt man \textit{homogene Wärmeleitunsgleichung}.	
\end{genericdf}

\begin{genericdf}{Schallausbreitung}
Durch
\begin{align*}
\partial^2_t u- c \ \gamma u = 0
\end{align*}
ist die \textit{homogene Wellengleichung} mit Ausbreitungsgeschwindigkeit $c > 0$ gegeben.
Weiter ist mit
\begin{align*}
v = \frac{1}{\rho} \nabla u
\end{align*}
die Geschwindigkeitsverteilung und mit
\begin{align*}
p = p_0 - \partial_t u
\end{align*}
die Druckverteilung gegeben.
Man wählt 
\begin{align*}
u(t,x) = e^{\i\lambda t} w(x)
\end{align*}
als Ansatz zur Entkopplung von Zeit und Ort.
Durch Einsetzen erhält man 
\begin{align*}
- \lambda^2 e^{\i\lambda t} w(x) - 
c e^{\i\lambda t} \Delta w(x) = 0
\Leftrightarrow
\Delta w(x)+ \frac{\lambda^2}{c} w(x) = 0
\end{align*}
die \textit{Helmholtz-Gleichung}.
\end{genericdf}

\begin{genericdf}{Quantenmechanik}
Es ist die Aufenthaltsgeschwindigkeit
\begin{align*}
\varphi : [0, \infty) \times \R^3 \to \R
\end{align*}
eines Elektrons gesucht.
Insbesondere muss
\begin{align*}
\int \limits_{\R^3} \varphi(t,x) \td{x}= 1
\end{align*}
erfüllt sein.
Man findet $\varphi$ durch Lösen der durch
\begin{align*}
\i \  \partial_t \varphi  = - \Delta \varphi
\end{align*}
gegebenen \textit{Schrödingergleichung}.
\end{genericdf}

\begin{genericdf}{Plattengleichung}
Die \textit{Plattengleichung} ist durch
\begin{align*}
\partial_t^2 - c \ \Delta^2 u = 0
\end{align*}
gegeben.
\end{genericdf}

\begin{genericdf}{Elektromagnetische Wellen}
Gegeben sei die Stromdichte 
$j : [0, \infty) \times \R^3 \to \R^3$
und die Ladungsdichte
$\rho : [0,\infty) \times \R^3 \to \R$.
Gesucht ist nun die elektrische/magnetische Feldstärke $E,H : [0, \infty)\times \R^3 \to \R^3$, 
welche die durch
\begin{align*}
\partial_t E + j &= \nabla \times H\\
\partial_t H 	&= - \nabla \times E\\
\nabla \cdot E &= \rho\\
\nabla \cdot H &= 0
\end{align*}
gegeben \textit{Maxwell-Gleichungen} erfüllen.
Durch 
\begin{align*}
\Delta E &= \nabla( \nabla \cdot E ) - \nabla \times ( \nabla \times E) 
= \nabla \rho + \nabla \times \partial_t H 
= \nabla \rho + \partial_t(\nabla \times  H)\\
&= \nabla \rho + \partial_t(\partial_t E + j) 
\end{align*}
folgt mit
\begin{align*}
\partial^2_t E = \Delta E - \partial_t j - \nabla \rho
\end{align*}
die \textit{inhomogene Wellengleichung}
\end{genericdf}

\newpage
\section{Typeinteilung}

\begin{df}

Sei die lineare partielle Differentialgleichung 
$2.$ Ordnung gegeben durch
\begin{align}\label{equation:1.2.1}
\sum \limits_{i,k= 0}^n a_{ik}(x) \ \partial_{x_i} \partial_{x_k} u(x)
+ \sum \limits_{i = 0}^n b_i(x) \ \partial_{x_i} u(x)
+ c(x) \ u(x).
\end{align}
Dabei ist $A(x) = (a_{ik}(x))_{i,k = 0,\dots,n}$ eine reelle symmetrische Matrix und es gilt\\
$x = (x_0, x_1, \dots, x_n) \in [0,\infty) \times \R^n$.\\
Seien $\lambda_1(x), \dots, \lambda_n(x)$ die nicht notwendigerweise verschiedenen Eigenwerte von $A(x)$.
Die Eigenwerte werden entsprechend ihrer Vielfachheit gezählt.
Dann nennt man
\begin{align*}
d(x) = \# \lbrace j \in \lbrace 0, \dots, n \rbrace \ | \ \lambda_j(x) = 0 \rbrace
\end{align*}
die \bi{Defektzahl} und 
\begin{align*}
t(x) = \# \lbrace j \in \lbrace 0, \dots, n \rbrace \ | \ \lambda_j(x) < 0 \rbrace
\end{align*}
den \bi{Trägheitsindex}.
Nun heißt die Differentialgleichung \ref{equation:1.2.1} im Punkt $x$
\renewcommand{\labelenumi}{(\roman{enumi})}
\begin{enumerate}
\item \bi{hyperbolisch}, falls $d(x)= 0 $ und $t(x) \in \lbrace 1, n \rbrace$.

\item \bi{parabolisch}, falls $d(x) > 0$.

\item \bi{elliptisch}, falls $d(x)  = 0$ und $t(x) \in \lbrace 0, n+1 \rbrace$.

\item \bi{ultrahyperbolisch}, falls $d(x) = 0 $ und $1 < t(x) < n$.
\end{enumerate}
\end{df}

\begin{genericdf}{Beispiele}
\renewcommand{\labelenumi}{(\alph{enumi})}
\begin{enumerate}
\item 
Für die Wellengleichung 
\begin{align*}
\partial^2_t u - \Delta u = 0
\end{align*}
ist
\begin{align*}
(a_{ik}) = \diag(1,-1, \dots, \-1)
\end{align*}
unabhängig von $x$.
Damit ist $t(x) = n$ und $d(x) = 0$.
Also ist die Wellengleichung hyperbolisch.

\item
Für die Laplace-bzw. Poissiongleichung
\begin{align*}
\Delta u = 0, \quad \Delta u = f
\end{align*}
gilt
\begin{align*}
(a_{ik}) = \Id = \diag(1, \dots, 1),
\end{align*}
woraus $d(x) = t(x) = 0$ folgt.
Damit sind beide Gleichungen elliptisch.

\item
Genauso wie in (b) erkennt man, dass die Helmholtz-Gleichung
\begin{align*}
\Delta u + c \ u = 0
\end{align*}
elliptisch ist.

\item
Die Wärmeleitunsgleichung 
\begin{align*}
\partial_t u - \Delta u  = 0 
\end{align*}
ist mit 
\begin{align*}
(a_{ik}) = \diag(0,-1, \dots , -1)
\end{align*}
parabolisch ($d(x) = 1$).

\item
Die Tricomi-Gleichung
\begin{align*}
x_2 \partial_{x_1}^2 u + \partial_{x_2}^2 u = 0
\end{align*}
mit 
\begin{align*}
a_{ik}(x_1,x_2) = 
\begin{pmatrix}
x_2 & 0 \\
0  & 1 
\end{pmatrix}
\end{align*}
ist 
\begin{itemize}
\item elliptisch für $x_2 > 0$.
\item parabolisch für $x_2 = 0$.
\item hyperbolisch für $x_2 < 0$.
\end{itemize}
\end{enumerate}
\end{genericdf}

\begin{genericdf}{Fragen}
\renewcommand{\labelenumi}{\theenumi.}
\begin{enumerate}
\item Existenz von Lösungen.
\item Berechnung oder Konstruktion von Lösungen.
\item Welche Bedingungen sind für die Eindeutigkeit zu stellen.
\item Stetige Abhängigkeit der Lösung von gegebenen Daten.
\end{enumerate}
\end{genericdf}
\chapter{Ganzraumprobleme}
\section{Die Laplace-Poisson Gleichung}

\begin{df}
Man nennt $u \in C^2( \Omega \to \R)$
\bi{harmonisch}, falls
\begin{align*}
\Delta u = 0
\end{align*}
in $\Omega$ erfüllt ist.
\end{df}
Nun sucht man eine radialsymmetrische harmonische Funktion
$u = u(|x|) = u(r)$.
Den Laplace-Operator kann man durch
\begin{align*}
\Delta
= \frac{\partial^2}{\partial r^2} 
+ \frac{m-1}{r} \ \frac{\partial}{\partial r}
+ \text{Terme mit Ableitung nach Winkeln}
\end{align*}
in Polarkoordinaten im $\R^m$ schreiben.
Damit erhält man mit
\begin{align*}
u^{\prime \prime}(r) + \frac{m-1}{r} \ u^\prime(r) = 0
\end{align*}
das Fundamentalsystem
\begin{align*}
u(r) = c, \quad
u(r) =
\begin{cases}
\frac{c}{r^{m-2}}, &\ \text{falls} \ m\geq 3\\
c \ \ln(r) , &\ \text{falls} \ m = 2
\end{cases}
\end{align*}
der Laplace-Gleichung.

\begin{df}
Die \bi{Fundamentallösung} der Laplacegleichung ist durch
\begin{align*}
\Phi_m(r)
:= 
\begin{cases}
\frac{1}{2 \pi} \ \ln(r), &\ \text{falls} \ m=2 \\
\frac{-1}{(m-2) \ |S_{m-1}| r^{m-2}}, &\ \text{falls} \ m \geq 3
\end{cases}
\end{align*}
gegeben.
Dabei ist $|S_{m-1}|$ das Maß der Einheitssphäre in $\R^m$.
Es gilt beispielsweise $|S_1| = 2 \Pi$.
\end{df}
\end{document}