\section{Die Laplace-Poisson Gleichung}

\begin{df}
Man nennt $u \in C^2( \Omega \to \R)$
\bi{harmonisch}, falls
\begin{align*}
\Delta u = 0
\end{align*}
in $\Omega$ erfüllt ist.
\end{df}
Nun sucht man eine radialsymmetrische harmonische Funktion
$u = u(|x|) = u(r)$.
Den Laplace-Operator kann man durch
\begin{align*}
\Delta
= \frac{\partial^2}{\partial r^2} 
+ \frac{m-1}{r} \ \frac{\partial}{\partial r}
+ \text{Terme mit Ableitung nach Winkeln}
\end{align*}
in Polarkoordinaten im $\R^m$ schreiben.
Damit erhält man mit
\begin{align*}
u^{\prime \prime}(r) + \frac{m-1}{r} \ u^\prime(r) = 0
\end{align*}
das Fundamentalsystem
\begin{align*}
u(r) = c, \quad
u(r) =
\begin{cases}
\frac{c}{r^{m-2}}, &\ \text{falls} \ m\geq 3\\
c \ \ln(r) , &\ \text{falls} \ m = 2
\end{cases}
\end{align*}
der Laplace-Gleichung.

\begin{df}
Die \bi{Fundamentallösung} der Laplacegleichung ist durch
\begin{align*}
\Phi_m(r)
:= 
\begin{cases}
\frac{1}{2 \pi} \ \ln(r), &\ \text{falls} \ m=2 \\
\frac{-1}{(m-2) \ |S_{m-1}| r^{m-2}}, &\ \text{falls} \ m \geq 3
\end{cases}
\end{align*}
gegeben.
Dabei ist $|S_{m-1}|$ das Maß der Einheitssphäre in $\R^m$.
Es gilt beispielsweise $|S_1| = 2 \Pi$.
\end{df}