\section{Partielle Differentialgleichungen aus der Physik}
\begin{genericdf}{Elektrostatik}
Sei $\rho  :  \R^3 \to \R$ eine Ladungsverteilung bzw. Dichte
und $E : \R^3 \to \R^3$ ein elektrisches Feld.
Aus der Physik ist der Zusammenhang
\begin{align*}
\underbrace{\int_\Omega \rho \td{x}}_{\text{Ladung in $\Omega$}}
=
\underbrace{\int_{\partial\Omega} n \ E \td{\sigma}}_{\text{Aus $\Omega$ ausfließendes Feld}}
\stackrel{\text{Gauß-O.}}{=}
\int_\Omega \div E \td{x}
\end{align*}
für \glqq alle\grqq~Gebiete $\Omega \subseteq \R^3$ bekannt.
Hierbei ist $n$ der aus $\Omega$ hinauszeigende Normaleneinheitsvektor.
Da $\Omega$ beliebig ist, folgt
\begin{align*}
\rho = \div E 
:= \partial_{x_1} E_1 + \partial_{x_2} E_2 + \partial_{x_3} E_3.
\end{align*}
Außerdem ist ein elektrisches Feld wirbelfrei, womit
\begin{align*}
\rot E := \nabla \times E
=
\begin{pmatrix}
\partial_{x_1} \\
\partial_{x_2} \\
\partial_{x_3}
\end{pmatrix} 
\times
\begin{pmatrix}
E_1 \\ E_2 \\ E_3
\end{pmatrix}
= 
\begin{pmatrix}
\partial_{x_2} E_3 - \partial_{x_3} E_2\\
\partial_{x_3} E_1 - \partial_{x_1} E_3\\
\partial_{x_1} E_2 - \partial_{x_2} E_1
\end{pmatrix}
= 
0
\end{align*}
gilt. Damit besitzt ein $E$ ein Potential $\varphi : \R^3 \to \R$, d.h. es gilt
\begin{align*}
E = \grad \varphi := \nabla \varphi
:=
\begin{pmatrix}
\partial_{x_1} \varphi \\
\partial_{x_2} \varphi \\
\partial_{x_1} \varphi 
\end{pmatrix} 
\Rightarrow
\Delta \varphi = \rho.
\end{align*}
Dabei bezeichnet
\begin{align*}
\Delta = \partial_{x_1}^2 + \partial_{x_2}^2 + \partial_{x_3}^2
\end{align*}
den \textit{Laplace-Operator}.
Nun bezeichnet man
\begin{align*}
\Delta \varphi &= \rho\\
\Delta \varphi &= 0
\end{align*}
als \textit{Poisson-und Laplacegleichung}.
\end{genericdf}