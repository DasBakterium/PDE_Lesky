\section{Partielle Differentialgleichungen aus der Physik}
\begin{genericdf}{Elektrostatik}
Sei $\rho  :  \R^3 \to \R$ eine Ladungsverteilung bzw. Dichte
und $E : \R^3 \to \R^3$ ein elektrisches Feld.
Aus der Physik ist der Zusammenhang
\begin{align*}
\underbrace{\int_\Omega \rho \td{x}}_{\text{Ladung in $\Omega$}}
=
\underbrace{\int_{\partial\Omega} n \ E \td{\sigma}}_{\text{Aus $\Omega$ ausfließendes Feld}}
\stackrel{\text{Gauß-O.}}{=}
\int_\Omega \div E \td{x}
\end{align*}
für \glqq alle\grqq~Gebiete $\Omega \subseteq \R^3$ bekannt.
Hierbei ist $n$ der aus $\Omega$ hinauszeigende Normaleneinheitsvektor.
Da $\Omega$ beliebig ist, folgt
\begin{align*}
\rho = \div E 
:= \partial_{x_1} E_1 + \partial_{x_2} E_2 + \partial_{x_3} E_3.
\end{align*}
Außerdem ist ein elektrisches Feld wirbelfrei, womit
\begin{align*}
\rot E := \nabla \times E
=
\begin{pmatrix}
\partial_{x_1} \\
\partial_{x_2} \\
\partial_{x_3}
\end{pmatrix} 
\times
\begin{pmatrix}
E_1 \\ E_2 \\ E_3
\end{pmatrix}
= 
\begin{pmatrix}
\partial_{x_2} E_3 - \partial_{x_3} E_2\\
\partial_{x_3} E_1 - \partial_{x_1} E_3\\
\partial_{x_1} E_2 - \partial_{x_2} E_1
\end{pmatrix}
= 
0
\end{align*}
gilt. Damit besitzt ein $E$ ein Potential $\varphi : \R^3 \to \R$, d.h. es gilt
\begin{align*}
E = \grad \varphi := \nabla \varphi
:=
\begin{pmatrix}
\partial_{x_1} \varphi \\
\partial_{x_2} \varphi \\
\partial_{x_1} \varphi 
\end{pmatrix} 
\Rightarrow
\Delta \varphi = \rho.
\end{align*}
Dabei bezeichnet
\begin{align*}
\Delta = \partial_{x_1}^2 + \partial_{x_2}^2 + \partial_{x_3}^2
\end{align*}
den \textit{Laplace-Operator}.
Nun bezeichnet man
\begin{align*}
\Delta \varphi &= \rho\\
\Delta \varphi &= 0
\end{align*}
als \textit{Poisson-bzw. Laplacegleichung}.
\end{genericdf}

\begin{genericdf}{Wärmeleitung}
Sei $\nu:[0, \infty) \times \R^3 \to \R$ eine Temperaturverteilung.
Dann ist die Wärmeenergie durch
\begin{align*}
\int \limits_\Omega c \cdot \nu(t,x) \td{x}
\end{align*}
für eine Materialkonstante $c$ gegeben.
Die Energieänderung erhält man nun durch
\begin{align*}
\int \limits_\Omega c \cdot \nu(t_1,x) \td{x}
&-
\int \limits_\Omega c \cdot \nu(t_2,x) \td{x}
\stackrel{\text{Physik}}{=}
\int \limits_{t_1}^{t_2} 
\underbrace{\int \limits _{\partial \Omega} 
\alpha \ \nabla \nu(t,x) \cdot n \td{x}}_{\text{Wärmefluss aus $\Omega$}} \td{t}\\
&\stackrel{\text{Gauß-O.}}{=}
\int \limits_{t_1}^{t_2}
\int \limits_\Omega 
\nabla \cdot ( \alpha \ \nabla \nu(t,x)) \td{x} \td{t}.
\end{align*}
Da $t_1,t_2 $ und $\Omega$ beliebig war, folgt
\begin{align*}
\int \limits_{\Omega}
c \ \partial_t \nu(t,x)  \td{x}
= 
\int \limits_{\Omega}
\nabla \cdot ( \alpha \ \nabla \nu(t,x)) \td{x}
\end{align*}
und
\begin{align*}
\partial_t \nu(t,x) = \frac{1}{c} \ \nabla \cdot ( \alpha \ \nabla \nu(t,x)).
\end{align*}
Es folgt
\begin{align*}
\partial_t \nu(t,x) = \frac{\alpha}{c} \Delta\nu(t,x),
\end{align*}
falls $\alpha$ eine Konstante ist.
Diese Differentialgleichung nennt man \textit{homogene Wärmeleitunsgleichung}.	
\end{genericdf}

\begin{genericdf}{Schallausbreitung}
Durch
\begin{align*}
\partial^2_t u- c \ \gamma u = 0
\end{align*}
ist die \textit{homogene Wellengleichung} mit Ausbreitungsgeschwindigkeit $c > 0$ gegeben.
Weiter ist mit
\begin{align*}
v = \frac{1}{\rho} \nabla u
\end{align*}
die Geschwindigkeitsverteilung und mit
\begin{align*}
p = p_0 - \partial_t u
\end{align*}
die Druckverteilung gegeben.
Man wählt 
\begin{align*}
u(t,x) = e^{\i\lambda t} w(x)
\end{align*}
als Ansatz zur Entkopplung von Zeit und Ort.
Durch Einsetzen erhält man 
\begin{align*}
- \lambda^2 e^{\i\lambda t} w(x) - 
c e^{\i\lambda t} \Delta w(x) = 0
\Leftrightarrow
\Delta w(x)+ \frac{\lambda^2}{c} w(x) = 0
\end{align*}
die \textit{Helmholtz-Gleichung}.
\end{genericdf}

\begin{genericdf}{Quantenmechanik}
Es ist die Aufenthaltsgeschwindigkeit
\begin{align*}
\varphi : [0, \infty) \times \R^3 \to \R
\end{align*}
eines Elektrons gesucht.
Insbesondere muss
\begin{align*}
\int \limits_{\R^3} \varphi(t,x) \td{x}= 1
\end{align*}
erfüllt sein.
Man findet $\varphi$ durch Lösen der durch
\begin{align*}
\i \  \partial_t \varphi  = - \Delta \varphi
\end{align*}
gegebenen \textit{Schrödingergleichung}.
\end{genericdf}

\begin{genericdf}{Plattengleichung}
Die \textit{Plattengleichung} ist durch
\begin{align*}
\partial_t^2 - c \ \Delta^2 u = 0
\end{align*}
gegeben.



\end{genericdf}